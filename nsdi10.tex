\documentclass{sig-alternate}
%\documentclass[10pt]{article}
%\clubpenalty=10000
%\widowpenalty = 10000
%\usepackage[margin=.9in]{geometry}
\usepackage[tight]{subfigure}

%\setlength\topmargin{0in}
%\setlength\headheight{0in}
%\setlength\headsep{0in}
%\setlength\oddsidemargin{0in}
%\setlength\evensidemargin{0in}
%\setlength\parindent{0.25in}
%\setlength\parskip{0in}

\usepackage{url}
\usepackage{multirow}
\usepackage{array}
\usepackage{epsfig}
\usepackage{footnote}
\widowpenalty=10000
\clubpenalty=10000
%\usepackage{setspace}
%\doublespacing

\begin{document}

\title{On the Design and Implementation of Structured P2PVPNs}

\numberofauthors{2}
\author{
\alignauthor
{David Isaac Wolinsky, Kyungyong Lee, Yonggang Liu, P. Oscar Boykin,
Renato Figueiredo}\\
       \affaddr{University of Florida}\\
       \email{\{davidiw, klee, yonggang, boykin, renato\}@acis.ufl.edu}
\alignauthor
{Linton Abraham}\\
  \affaddr{Clemson University}\\
  \email{labraha@clemson.edu}
}

\maketitle
\begin{abstract}
\begin{sloppypar}
In recent years, P2P VPNs have become quite popular by allowing users to connect
directly with each other bypassing the overhead of communicating through a third
party proxy. These P2P VPNs require connecting to a central server for
authentication, NAT traversal, and proxying in the off chance NAT traversal
fails. This significantly improves upon classical, centralized VPNs, though it
adds a new complexity either maintenance of all-to-all connections during
run-time or the involvement of a centralized authentication entity for each live
connection attempt.  For this solution, we propose a completely run-time
decentralized P2P model based upon a structured P2P system.  In this paper, we
will describe the components of this model as well as present and evaluate our
reference implementation. A decentralized P2PVPN has an intuitive and simplistic
setup, reduces the requirements for connectivity, offers better proxy selection
in lieu of NAT traversal, and provides an opportunity for more intuitive trust
solutions. For evaluation, we will compare system and networking overheads of
the different VPN technology focusing on latency, bandwidth, CPU, and memory.
\end{sloppypar}
\end{abstract}

\section{Introduction}
A Virtual Private Network (VPN) provides the illusion of a Local Area Network
(LAN), namely direct communication, over a wide area network such as the
Internet while guaranteeing secure and authenticated communication amongst
participants.  Common uses of VPNs include accessing company or academic
network resources while traveling abroad, playing LAN based video games over the
Internet, connecting distributed resources from multiple sites, and securing
your Internet traffic while in unsecure locations.  In the context of this
paper, we focus on VPNs that provide connectivity between individual resources
and so all resources that need symmetric connectivity will need to be
configured wth VPN software.

While traditional VPNs enable such distributed connectivity they do so at the
cost of maintaining a central server, which becomes the conduit for all traffic,
becoming a performance bottleneck and potentially removing end-to-end security.
To alleviate this, there have been three directions 1) support for multiple VPN
servers for a single VPN~\cite{openvpn, cloudvpn}, 2) the use of P2P
connections for bypassing central communication that rely on run-time central
authentication \cite{hamachi, wippien}, and 3) the use of unstructured P2P
networks to form VPNs based upon shared secrets without user authentication and
limitations on network size~\cite{p2pvpn, n2n, tinc}.  In this paper, we present
a novel approach to forming secure, scalable, efficient, and self-configuring VPNs
through the the use of Structured P2P systems that has no reliance on
centralized systems after initialization.  Structured P2P technology enables
users to communicate directly with all users without knowing anything beyond
their virtual IP bypassing the need for centralization while providing
all-to-all communication without maintaining all-to-all connectivity with
participants.  Interesting applications of P2P include efficient wide area
multicast, data distribution, storage, chat applications, and even IP
connectivity.

Current generation P2PVPNs do not scale well, provide features such as
full-tunneling of network traffic, such as forwarding Internet traffic, nor do
they have intuitive ability for scalable multicast or broadcast.  P2PVPNs rely
on direct connectivity and in general will not work if NAT (Network Address
Translation) traversal between peers is unsuccessful.  Unstructured P2P based
VPNs have similar issues, though have the ability to reuse the unstructured
overlay to relay packets though typically relying on all-to-all connectivity
in the system.  Furthermore, unstructured P2P systems currently lack the ability
to police participants in the system.

The problems we seek to address with our P2PVPN model include:
\small {
\begin{itemize}
\setlength{\itemsep}{0pt}
\setlength{\parskip}{0pt}
\item reducing the role of centralization for user authentication in a VPN
\item managing participants in a live system
\item supporting full-tunneling of Internet traffic in a P2P system
\item handling relay selection in lieu of unsuccessful NAT traversal
\item supporting multicast and broadcast communication
\end{itemize}
}
A rudimentary overview of our solutions to the above problems follows and will
be covered in depth in the rest of this paper.  To provide fully decentralized
run-time connectivity and policing, we use an automated certificate authority based
upon the use of user groups.  In the case of full-tunneling, P2PVPNs introduce
significantly more complexity since a simple routing table swap as done in
central VPNs no longer work, as such we investigate three different mechanisms
for tunneling all Internet traffic to our full-tunnel endpoint(s) besides our
P2P traffic.  When nodes cannot directly communicate, they seek to connect to
peers that are mutually physically close to each other and use them to relay
communication.  For efficient multicast and broadcast communication, we rely on
the use of bootstrapping a private P2P system whose members are only
participants of the VPN.

Explicitly, our contributions made in this paper are:
\small{
\begin{itemize}
\setlength{\itemsep}{0pt}
\setlength{\parskip}{0pt}
\item automated group-based certificate authority
\item three different approaches to configuring full-tunneling
\item intelligent selection of relays
\item use of a private P2P VPN system bootstrapped of a general P2P system
\end{itemize}
}

The rest of this paper is organized as follows.  Section II gives an overview
of current VPN technologies and the efforts to decentralized.  Section III
introduces P2P structures and our previous work IPOP (IP over P2P).  Section IV
describes the contributions of this paper, namely a feature-full P2PVPN.  In
Section V, we discuss our implementation and present evaluation comparing
centralized, P2P, and our VPN.  Finally, we give some concluding remarks in
Section VI.

\section{Virtual Private Networks}
There exist many different flavors of virtual networking, this paper focuses on
those that are used to create or extend a virtual layer 3 network.  A few
examples of such technologies include Cisco's Systems VPN and AnyConnect VPN
Client~\cite{cisco} as well as OpenVPN~\cite{openvpn}.  In this section, we
begin by going in depth on client configuration of VPNs, overview concepts of
server roles in centralized VPNs,  and then roles of participants in P2P VPNs.
Finally we conclude the section by presenting a table~\ref{tab:vpns} which
compares qualitatively the features of VPNs that fit in these categories.

\subsection{Client VPN Configuration}
In figure~\ref{fig:vpn}, we abstract the common features of all VPNs with focus
on the client.  The key components of the client are 1) client software that
communicates with the VPN overlay directly and 2) a virtual network device.
During initialization VPN software begins by authenticating with some overlay
agent, optionally it then queries the agent for information about the network
such as the network address space, and then starts the virtual network device.

There are many different mechanisms for communicating with the overlay agent.
For quick setup, a system may provide a shared secret password or key that is
common for the entire network.  A more user-friendly and manageable approach
re-uses the shared secret mechanism and then adds user accounts and passwords,
thus blocking unauthorized users from the VPN, while still making it somewhat
difficult for brute force attacks to work, so long as the key remains private.
For the strongest level of security, each client can be configured to have a
signed-certificate that makes brute force attacks all but impossible.  The
tradeoffs come in terms of usability.  While the use of uniquely
sign-certificates may be the most secure, it can be quite difficult for novice
computer users.  A good balance found in many environments is the mixture of
a shared secret and user account, where the shared secret is included with
the installatoin of the VPN application where the application is distributed
from a secured site.

Once the user has connected with the overlay, the virtual networking device
needs to be configured enabling the user's machine to communicate with other
participants in the VPN.  This configuration varies by VPN, commonly though,
this information contains the network address space, an allocation of an
address for the user's machine, and potentially a remote peer for
full-tunneling.

In order to communicate over the VPN transparently, there must exist a network
device driver that allows common network APIs such as Berkeley Sockets and
hence existing application to work without modification.  There are many
different types of virtual networking devices, though due to our focus
on an open platform, we focus on TAP~\cite{tap}. TAP allows the creation of one
or more Virtual Ethernet and / or IP devices and is available for almost all
modern operating systems including Windows, Linux, Mac OS/X, BSD, and Solaris.
A TAP device exists as a file descriptor providing read and write operations. 
Incoming packets from the virtual network are written to the TAP device and the
networking stack in the OS delivers the packet to the appropriate socket. 
Packets that are read from a TAP device are those that are sent by sockets to
the virtual network.

The virtual network device is either configured using static addressing or
dynamically through dynamic host configuration process (DHCP)
\cite{dhcp0, dhcp1}.  This causes a new routing rule that causes all packets
sent to the virtual network address space to be sent to the virtual network
device.  When a packet is read from the TAP device, it can then be sent to
the overlay via the client application.  The overlay will deliver the packet
to another end point, which can be a client or a server enabled with virtual
networking stack.  When receiving a packet, it will be written to the TAP
device.  In most cases, the IP layer header will remain unchanged while
configuration will determine how the Ethernet header will be handled.

The described configuration so far, creates what is known as a split tunnel,
or virtual network connection that only has passes traffic directly related
to the virtual network and not Internet traffic.  Another form of tunneling
exists called full tunneling.  Full tunneling allows a VPN client to securely
forward all their Internet traffic through a VPN router.  This enables a user
to ensure all their Internet communication originates from a secure and trusted
location and provides some level of security when a user is an insecure and
potentially hostile environment, such as an open wireless network at coffee shop.

Most centralized VPNs implement full-tunneling by doing a routing rule swap,
where the default gateway becomes an end-point in the VPNs subnet and traffic
for the VPN server is routed to the local network gateway.  For example, on a
typical home network, all traffic for the VPN server is sent via to the networks
router and then via the Internet to the VPN server.  All other traffic is sent
to the virtual network device, then sent securely to the VPN server.  In a P2P
system, there becomes two new considerations 1) P2P traffic must not be routed
to the VPN gateway and 2) there may be more than one VPN gateway.  Allowing
more than one VPN gateway per VPN allows distribution of the cost of maintaining
a full-tunnel who will have 3 additional messages for each incoming packet.
We discuss and provide solutions to this problem in \ref{fulltunnel}.

\subsection{Centralized VPN Servers}

\subsection{Centralized P2P VPN Systems}

\subsection{P2PVPN Client / Server Roles}
Unlike centralized systems, pure P2P systems have no concept of dedicated
servers, not to say that a user cannot start an instance of the P2P VPN software
purely for enhancing or enabling connecctivity.  In these systems, all
participants are members of a collective known as an overlay.  Current generation
P2P, decentralized VPNs use a P2P unstructured network, where there are no
guarantees about distance and routability between peers.  As a result
participants tend to be connected to a random distribution of peers in the
overlay.  Finding a peer requires either global knowledge of the pool or at
worst case broadcasting a lookup message to the entire overlay.  While
unstructured P2P systems have some scalability concerns, P2P systems in general
allow for server-less systems.  In the realm of VPNs, all client VPNs are also
servers with varying different responsibilities depending on the VPN
application, as we present in table~\ref{vpns}.

Typically, decentralized, P2P VPNs begin by attempting to connect to well known
end points running the P2P overlay, a list of such end points is distributed
with the application or some other out-of-band mechanism.  In the case of
P2PVPN, this involves communication with one or more BitTorrent trackers to
find other members of the P2PVPN group.  N2N requires knowledge of any
existing peer in the system.  It uses this endpoint to bootstrap more
connections to other peers in the system, allowing the application to be an
active participant in the overlay and potentially be a bootstrap connection for
other peers attempting to connect.

\small{
\begin{table*}[ht]
\setlength{\itemsep}{0pt}
\setlength{\parskip}{0pt}
\centering
%\begin{tabular}[c]{|p{1.1cm}||p{3.475cm}|p{3.475cm}|p{3.475cm}|p{3.475cm}|} \hline
\begin{tabular}[c]{|p{1.5cm}||p{2.7cm}|p{2.7cm}|p{2.7cm}|p{2.7cm}|p{2.7cm}|p{2.7cm}|} \hline
& VPN Type & Authentication Method & Peer Discovery & NAT Traversal & Availability \\ \hline

OpenVPN & Centralized with multiple servers & Certificates or passwords with a central server &
Stored at central server(s) & Relay through server(s) & Open Source\\ \hline

CloudVPN & Centralized with multiple servers & CA-signed Certificates & Broadcast &
Relay through server(s) & Open Source\\ \hline

Hamachi & Centralized P2P & Password at central server & Stored at central server &
NAT traversal and relay through central server & personal use only, no private servers \\ \hline

GBridge & Centralized P2P & Password at central server & Stored locally & NAT traversal and
relay through central server & free to use, close source, no private relays, Windows only \\ \hline

Wippien & Centralized P2P & Password at central server & Stored locally & NAT traversal,
no relay support & Mixed Open / Closed source\\ \hline

N2N & Unstructured P2P & Shared secret & Broadcast look up & NAT traversal and support of relay through
overlay & Open Source \\ \hline

P2PVPN & Unstructured P2P & Shared secret & Everyone knows about everyone else & No NAT traversal,
support of relay through overlay & Open Source \\ \hline

tinc & Unstructured P2P & CA Certificates / Private key & Everyone knows about
everybody  & No NAT traversal, support of relay through overlay & Open Source\\ \hline

IPOP & Structured P2P & CA Certificates or pre-exchanged keys &
DHT lookup & NAT traversal and relay through physically close peers &
Open Source\\ \hline

\end{tabular}
\caption{VPN Comparison}
\end{table*}
}

\section{Structured Peer-to-Peer Systems}
\section{Components of a P2PVPN}
\label{p2pvpn}
\subsection{Full-Tunneling over P2P}
\label{fulltunnel}

\section{Evaluating VPN Models}
\section{Conclusions}

\bibliographystyle{abbrv}
\bibliography{nsdi10}
\suppressfloats
\end{document}
